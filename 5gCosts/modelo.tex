\documentclass[t]{beamer}
\usepackage[portuges]{babel}
\usepackage[utf8]{inputenc}
\usepackage[T1]{fontenc}
\usepackage{amsmath}
\usepackage{amssymb}
\usepackage{amsfonts}
\usepackage{amsthm}
\usepackage{graphicx}
\usepackage{xcolor}
\usepackage[scaled]{helvet}
\renewcommand{\familydefault}{\sfdefault}

%%%
%%% Define cores
%%%
\definecolor{cinza}{HTML}{75818B}

%%%
%%% Remove a barra de navegação do Beamer
%%%
\setbeamertemplate{navigation symbols}{}

%%%
%%% Margem dos slides
%%%
\setbeamersize{text margin left=10mm,text margin right=5mm} 

%%%
%%% Redefine a fonte do título dos slides
%%%
\setbeamercolor{frametitle}{fg=cinza}
\setbeamerfont{frametitle}{series=\bfseries}
\setbeamerfont{frametitle}{size=\huge}

%%%
%%% Ajusta a posição do título dos slides e início do texto
%%%
\addtobeamertemplate{frametitle}{\vspace*{2mm}}{\vspace*{5mm}}

%%%
%%% Adiciona páginação nos slides
%%%
%%% Caso não queira, basta comentar este bloco inteiro
%%% para ocultar a paginação
%%%
\addtobeamertemplate{navigation symbols}{}{
\usebeamerfont{footline}
\usebeamercolor[fg]{footline}
}
\setbeamercolor{footline}{fg=cinza}
\setbeamerfont{footline}{series=\bfseries}
\setbeamerfont{footline}{size=\tiny}
\setbeamertemplate{footline}{
\usebeamerfont{page number in head}
\usebeamercolor[fg]{page number in head}
\hspace{5mm}
\insertframenumber/\inserttotalframenumber
\vspace{5mm}
}

%%%
%%% Redefine símbolo padrão do itemize
%%%
\setbeamertemplate{itemize items}[ball]

%%%
%%% Insere numeração nas figuras
%%%
\setbeamertemplate{caption}[numbered]

%%%
%%% Imagem de fundo a ser usada em todos os slides (exceto
%%% no primeiro e no último)
%%%
\usebackgroundtemplate
{
\includegraphics[width=\paperwidth,height=\paperheight]{fundo.png}
}

%%%
%%% Adiciona slide de "Estrutura"
%%%
\AtBeginSection[]{\frame{\frametitle{Estrutura}\tableofcontents
[current]}}

%%%
%%% Define fontes e cores do slide de "Estrutura"
%%%
\setbeamerfont{section in toc}{series=\bfseries}
\setbeamercolor{section in toc}{fg=gray}
\setbeamerfont{section in toc shaded}{series=\mdseries}
\setbeamercolor{section in toc shaded}{fg=gray!01}
\setbeamercolor{subsection in toc}{fg=cinza}
\setbeamercolor{subsection in toc shaded}{fg=gray!60}
\setbeamercolor{subsubsection in toc}{fg=cinza}
\setbeamercolor{subsubsection in toc shaded}{fg=gray!60}

\mode<presentation>
%%%
%%% Início
%%%
\begin{document}

%%%
%%% Slide da capa
%%%
{
\usebackgroundtemplate{\includegraphics[width=\paperwidth]{capa.png}}
\begin{frame}[plain]
\vspace{18mm}
%%%
%%% Título da Apresentação
%%%
\begin{flushright}
\textcolor{cinza}{\textbf{\huge{
Título da Apresentação
}}}
\end{flushright}

\vspace{-6mm}
%%%
%%% Nome do autor
%%%
\begin{flushright}
\textcolor{cinza}{\textbf{\scriptsize{
Nome do autor
}}}
\end{flushright}

\vspace{-7mm}
%%%
%%% Formação | Departamento | Centro
%%%
\begin{flushright}
\textcolor{cinza}{\scriptsize{
Formação | Departamento | Centro
}}
\end{flushright}


\end{frame}
}

%%%
%%% Demais slides (exceto o slide final)
%%%
\begin{frame}
\frametitle{Estrutura}
\tableofcontents
\end{frame}

\section{Seção 1}

\subsection{Subseção 1}

\begin{frame}
\frametitle{Título do texto}
Use este espaço para escrever o seu texto.
\end{frame}

\subsection*{Subseção 2}

\begin{frame}
\frametitle{Título do texto}
\begin{itemize}
\item Item a
\item Item b
\begin{itemize}
\item Item b.1
\item Item b.2
\end{itemize}
\item Item c
\end{itemize}
\end{frame}

\section{Seção 2}

\begin{frame}{Texto em duas colunas}
\begin{columns}[c]
\begin{column}{.5\textwidth}
\begin{figure}[!ht]
\centering
\includegraphics[scale=.2]{logo.png}
\caption{Legenda da imagem}
\label{fig:rotulo}
\end{figure}
\end{column}
\begin{column}{.5\textwidth}
Logo da UFSC na Figura \ref{fig:rotulo}.
\end{column}
\end{columns}
\end{frame}

%%%
%%% Slide final
%%%
{
\usebackgroundtemplate{\includegraphics[width=\paperwidth]{capa.png}}
\begin{frame}[plain]
\vspace{15mm}
\begin{center}
\textcolor{cinza}{
\textbf{Contato}
}
\end{center}
\vspace{-6mm}
\begin{center}
\textcolor{cinza}{\scriptsize{
E-mail: seunome@email.com
}}
\end{center}
\vspace{-6mm}
\begin{center}
\textcolor{cinza}{\scriptsize{
Telefone: (xx) xxxx-xxxx
}}
\end{center}
\vspace{-6mm}
\begin{center}
\textcolor{cinza}{\scriptsize{
Site: www.site.com.br
}}
\end{center}
\end{frame}
}

\end{document}

